% Caso de uso 1.4.3 -------------------------------------------------------------------------------
\begin{UseCase} {CU 1.4.3}{Eliminar proveedor}{
	Elimina el proveedor seleccionado de la base de datos.
}

% Información del caso de uso ---------------------------------------------------------------------

\UCitem{Versión}{1.0}
\UCccsection{Administración}
\UCccitem{Autor}{Gonzaga Aparicio Josue}
\UCccitem{Evaluador}{}
\UCccitem{Operación}{}
\UCccitem{Prioridad}{Alta}
\UCccitem{Complejidad}{Baja}
\UCccitem{Volatilidad}{}
\UCccitem{Madurez}{Baja}
\UCccitem{Estatus}{Edición}
\UCccitem{Fecha del último estatus}{22 noviembre}

% Revisión ----------------------------------------------------------------------------------------
% Copie y pegue este bloque tantas veces como revisiones tenga el caso de uso.
% Esta sección la debe llenar solo el Revisor

% Revisión Versión % Anote la versión que se revisó
\UCccsection{Revisión Versión 0.1 }

% Fecha % Anote la fecha en que se terminó la revisión
\UCccitem{Fecha}{} 

% Evaluador % Coloque el nombre completo de quien realizó la revisión
\UCccitem{Evaluador}{}

% Resultado % Coloque la palabra que mas se apegue al tipo de acción que el analista debe realizar
\UCccitem{Resultado}{}

% Observaciones % Liste los cambios que debe realizar el Analista.
\UCccitem{Observaciones}{}

% Atributos ---------------------------------------------------------------------------------------
	
\UCsection{Atributos}

	\UCitem{Actores}{
		\cdtRef{Actor:Empleado}{Empleado} y \cdtRef{Actor:Sistema}{Sistema}.
	}

	\UCitem{Propósito}{
		Actualizar la información de los proveedores cuando surjan cambios en sus datos.
	}

	\UCitem{Entradas}{
		
		\begin{UClist}			
			Información básica del proveedor:

			\UCli Nombre: (\cdtRef{proveedor:iapellido}{Primer apellido}, \cdtRef{proveedor:iiApellido}{Segundo apellido} y \cdtRef{proveedor:nombre}{Nombre}).
			\UCli \cdtRef{proveedor:Empresa}{Empresa}.
			\UCli \cdtRef{proveedor:Telefono}{Telefono}.
			\UCli \cdtRef{proveedor:Correo}{Correo de contacto}.
		\end{UClist}
	}

	\UCitem{Salidas}{
		Ninguna.
	}

	\UCitem{Precondiciones}{
		\begin{UClist}
			\UCli Que exista un registo previo.
		\end{UClist}
	}

	\UCitem{Postcondiciones}{
		Ninguna.
	}

	\UCitem{Reglas de negocio}{
		Ninguna.
	}

	\UCitem{Errores}{
		Ninguna.
	}

	\UCitem{Tipo}{
		
	}
\end{UseCase}

%Trayectoria principal ----------------------------------------------------------------------------

\begin{UCtrayectoria}
	
	\UCpaso [\UCactor]	En el la interfaz \cdtRef{UI}{Menú principal} selecciona \cdtButton{Menú proveedores}. 
	\UCpaso [\UCsist]		Muestra la interfaz \cdtRef{UI}{Menú provedores}. \label{CU 4.3:P2}
	\UCpaso [\UCactor]	Selecciona \cdtButton{Eliminar proveedor}.
	\UCpaso [\UCsist]		Accede a la base de datos. \refTray{A} \refTray{B}
	\UCpaso [\UCsist]		Muestra la interfaz \cdtRef{UI}{Eliminar proveedor}.
	\UCpaso [\UCactor]	Selecciona el proveedor que quiere eliminar.
	\UCpaso [\UCactor]	Muestra mensaje: \cdtRef{MSG}{¿Seguro que quiere eliminar el proveedor? Todos sus datos serán borrados}.
	\UCpaso [\UCactor]	Selecciona \cdtButton{Confirmar}. \refTray{C}
	\UCpaso [\UCsist]		Elimina proveedor de la base de datos.
	\UCpaso [\UCsist]		Muestra mensaje: \cdtRef{MSG}{Proveedor eliminado con éxito}.
	\UCpaso [\UCactor]	Selecciona \cdtButton{OK}
	\UCpaso [\UCsist]		Muestra la interfaz \cdtRef{UI}{Menú Proveedores}.

\end{UCtrayectoria}


% Trayectorias alternativas -----------------------------------------------------------------------

% Trayectoria A
\begin{UCtrayectoriaA}{A}{Sin proveedores previamente registrados}
	\UCpaso [\UCsist]		Muestra mensaje: \cdtRef{MSG}{No hay proveedores que eliminar}.
	\UCpaso [\UCactor]	Selecciona \cdtButton{OK}
	\UCpaso Regresa al paso \ref{CU 4.3:P2} de la trayectoria principal.
\end{UCtrayectoriaA}

% Trayectoria B
\begin{UCtrayectoriaA}{B}{Base de datos no disponible}
	\UCpaso [\UCsist]		Muestra mensaje: \cdtRef{MSG}{Base de Datos no disponible, por favor intente más tarde}.
	\UCpaso [\UCactor]	Selecciona \cdtButton{OK}
	\UCpaso Regresa al paso \ref{CU 4.3:P2} de la trayectoria principal.
\end{UCtrayectoriaA}

% Trayectoria C
\begin{UCtrayectoriaA}{C}{Selecciona Cancelar}
	\UCpaso [\UCactor]	Selecciona \cdtButton{Cancelar}
	\UCpaso Regresa al paso \ref{CU 4.3:P2} de la trayectoria principal.
\end{UCtrayectoriaA}