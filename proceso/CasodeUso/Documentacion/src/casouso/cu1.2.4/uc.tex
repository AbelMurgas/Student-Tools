\begin{UseCase}{CUTI5.1-8}{Registrar compra a proveedor} {%
	
El empleado registra la entrada de un producto al inventario.
	
}
	
\UCitem{Versión}{1.0}
\UCccsection{Administración}
\UCccitem{Autor}{Reyes Vaca Mauricio Alberto}
\UCccitem{Evaluador}{}
\UCccitem{Operación}{}
\UCccitem{Prioridad}{Alta}
\UCccitem{Complejidad}{Media}
\UCccitem{Volatilidad}{Baja}
\UCccitem{Madurez}{Alta}
\UCccitem{Estatus}{Edición}
\UCccitem{Fecha del último estatus}{20 de noviembre del 2021}

%% Copie y pegue este bloque tantas veces como revisiones tenga el caso de uso.
%% Esta sección la debe llenar solo el Revisor
% %--------------------------------------------------------
	%   % Revisión Versión (Anote la versión que se revisó.)
\UCccsection{Revisión Versión 0.1 }
% 	% FECHA: Anote la fecha en que se terminó la revisión.
\UCccitem{Fecha}{} 
% 	% EVALUADOR: Coloque el nombre completo de quien realizó la revisión.
\UCccitem{Evaluador}{}
% 	% RESULTADO: Coloque la palabra que mas se apegue al tipo de acción que el analista debe realizar.
\UCccitem{Resultado}{}
% 	% OBSERVACIONES: Liste los cambios que debe realizar el Analista.
\UCccitem{Observaciones}{
	\begin{UClist}
		%\RCitem{PC1}{\DONE{En la pantalla falta el botón del ojito}}{18 de Abril del 2018}
	\end{UClist}		
}

% %--------------------------------------------------------
	
\UCsection{Atributos}
	\UCitem{Actor(es)}{
		\cdtRef{Actor:RT}{Sistema} y \cdtRef{Actor:AT}{Administrador}
	}
	\UCitem{Propósito}{
		Agregar al sistema una nueva compra o proveedor.
	}
	\UCitem{Entradas}{

		\begin{UClist}

			Información básica de la compra a proveedor.			
			\UCli Nombre del proveedor: (\cdtRef{Alumno:primerApellido}{Primer apellido}, \cdtRef{Alumno:segundoApellido}{Segundo apellido} y \cdtRef{Alumno:nombre}{Nombre}).
			
			\UCli \cdtRef{tesis:Título de la empresa}{Nombre de la empresa}.

			\UCli \cdtRef{tesis:Producto}{Nombre del producto}

			\UCli \cdtRef{tesis:Cantidad}{Cantidad}

		\end{UClist}
	}
	\UCitem{Salidas}{

	}

	\UCitem{Precondiciones}{
		Tener al menos 1 proveedor registrado en el sistema.
	}
	\UCitem{Postcondiciones}{
		Ninguna
	}
	\UCitem{Reglas de negocio}{
		Ninguna
	}
	\UCitem{Errores}{
		Ninguna
	}
	\UCitem{Tipo}{Secundario}
\end{UseCase}

\begin{UCtrayectoria}
	
	\UCpaso[\UCactor] Da clic en el ícono \cdtButton{Menú} de la pantalla \cdtIdRef{IUTI5.1-2}{Index}.
	\UCpaso[\UCactor] Da clic en el ícono \cdtButton{Proveedor}.
	\UCpaso[\UCactor] Da clic en el ícono \cdtButton{Registrar compra a proveedor}.
	\UCpaso[\UCsist]  Muestra pantalla con formulario "Registrar compra a proveedor".
	\UCpaso[\UCactor] Llena el formulario con los datos básicos.
	\UCpaso[\UCactor] Da clic en el botón "Confirmar".
	\UCpaso[\UCsist] Verifica que todos los campos hayan sido llenados. \refTray{A}
	\UCpaso[\UCsist] Verifica que la cantidad ingresada sea válido de acuerdo con la regla de negocio RN1.\refTray{B}
	\UCpaso[\UCsist] Accede a la base de datos. \refTray{D}
	\UCpaso[\UCsist] Verifica que exista el proveedor. \refTray{C}
	\UCpaso[\UCsist] Guarda los cambios en la base de datos.
	\UCpaso[\UCsist] Envía mensaje 5: "La compra a proveedor ha sido registrado con éxito".
	\UCpaso[\UCsist] Vacía el formulario.
	\UCpaso[\UCsist] Muestra la pantalla \cdtIdRef{IUTI5.1-8}{Visualizar Resultado} con la información registrada.
	\UCpaso[\UCactor] Solcita terminar con el registro compra a proveedor presionando el botón \cdtButton{Regresar}.
	\UCpaso[\UCsist] Muestra la pantalla \cdtIdRef{IUTI5.1-2}{Index}.
	
\end{UCtrayectoria}


%...........::::::::::::TRAYECTORIAS ALTERNATIVAS::::::::::::::........
%-------------------------Trayectoria A------------------------------
\begin{UCtrayectoriaA}{A}{Campos incompletos}
	\UCpaso[\UCsist] Envía el mensaje 1: "Debes ingresar la información requerida en todos los campos.			
	\UCpaso[] Regresa al paso 5 de la trayectoria principal.
\end{UCtrayectoriaA}

%-------------------------Trayectoria B------------------------------
\begin{UCtrayectoriaA}{B}{La cantidad llenada es cero}
	\UCpaso[\UCsist] Envía el mensaje 2: "Debes ingresar un cantidad valida."			
	\UCpaso[] Regresa al paso 5 de la trayectoria principal.
\end{UCtrayectoriaA}

%-------------------------Trayectoria C------------------------------
\begin{UCtrayectoriaA}{C}{Proveedor previamente registrado}
	\UCpaso[\UCsist] Envía el mensaje 3: "No se encontro el proveedor que ingreso, intente registrarlo previo a esta operación."			
	\UCpaso[] Regresa al paso 5 de la trayectoria principal.
\end{UCtrayectoriaA}

%-------------------------Trayectoria D------------------------------
\begin{UCtrayectoriaA}{D}{Base de Datos caída}
	\UCpaso[\UCsist] Envía el mensaje 4: "Base de Datos no disponible, por favor intente más tarde".			
	\UCpaso[] Regresa al paso 5 de la trayectoria principal.
\end{UCtrayectoriaA}