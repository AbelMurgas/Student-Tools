\begin{UseCase}{CU 1.5.1}{Registrar producto nuevo al Inventario} {
	
	Registra en el sistema los datos de un nuevo producto a través de un formulario. Los campos a llenar son los siguientes:
	\UCli Nombre del producto (debe escribir el nombre del producto siguiendo la RN5.1)
	\UCli Código de Barras (debe escribir el código de barras siguiendo la RN5.2)
	\UCli Tipo (Debe seleccionar si es un medicamento, un abarrote o un equipo)
	\UCli Descripción del producto (Debe escribir la descripción del producto según la RN5.3)
	\UCli Proveedor (Del combobox debe seleccionar el proveedor del producto)
	\UCli Precio de venta (Escribir el precio de venta según la RN5.4)

}
	
\UCitem{Versión}{1.0}
\UCccsection{Administración}
\UCccitem{Autor}{Diego Armando Hernández Penilla}
\UCccitem{Evaluador}{}
\UCccitem{Operación}{}
\UCccitem{Prioridad}{Alta}
\UCccitem{Complejidad}{Baja}
\UCccitem{Volatilidad}{Baja}
\UCccitem{Madurez}{Alta}
\UCccitem{Estatus}{Edición}
\UCccitem{Fecha del último estatus}{04 de enero del 2022}

%% Copie y pegue este bloque tantas veces como revisiones tenga el caso de uso.
%% Esta sección la debe llenar solo el Revisor
% %--------------------------------------------------------
	%   % Revisión Versión (Anote la versión que se revisó.)
\UCccsection{Revisión Versión 0.1 }
% 	% FECHA: Anote la fecha en que se terminó la revisión.
\UCccitem{Fecha}{04 de enero del 2022} 
% 	% EVALUADOR: Coloque el nombre completo de quien realizó la revisión.
\UCccitem{Evaluador}{}
% 	% RESULTADO: Coloque la palabra que mas se apegue al tipo de acción que el analista debe realizar.
\UCccitem{Resultado}{}
% 	% OBSERVACIONES: Liste los cambios que debe realizar el Analista.
\UCccitem{Observaciones}{
	\begin{UClist}
		%\RCitem{PC1}{\DONE{En la pantalla falta el botón del ojito}}{18 de Abril del 2018}
	\end{UClist}		
}

% %--------------------------------------------------------
	
\UCsection{Atributos}
	\UCitem{Actor(es)}{
		\cdtRef{Actor}{Administrador de la farmacia}
	}
	\UCitem{Propósito}{
		Registrar en el sistema los datos de un nuevo producto.
	}
	\UCitem{Entradas}{
		\UCli Nombre del producto 
		\UCli Código de Barras
		\UCli Tipo
		\UCli Descripción del producto
		\UCli Proveedor
		\UCli Precio de venta
	}
	\UCitem{Salidas}{
		Ninguna.
	}

	\UCitem{Precondiciones}{
		Haber registrado al menos un proveedor.
	}
	\UCitem{Postcondiciones}{
		\UCli Modificar datos del producto.
		\UCli Resurtir existencias del producto.
		\UCli Agregar al carrito de venta para un cliente.
		\UCli Registrar una devolución por parte de un cliente.
	}
	\UCitem{Reglas de negocio}{
		\UCli RN-051
		\UCli RN-052
		\UCli RN-053
		\UCli RN-054.
	}
	\UCitem{Errores}{
		\UCli No hay proveedores registrados.
		\UCli El nombre del producto no es válido
		\UCli El código de barras no es válido
		\UCli El tamaño de la descripción del producto no es válido
		\UCli El precio de venta no es válido
		\UCli Faltan campos del formulario por llenar
		\UCli Hay registrado otro producto con el mismo código de barras.
	}
	\UCitem{Tipo}{
		Secundario ya que extiende el CU 1.3 Compra de productos a proveedor y el CU 1.5 Gestionar Inventario
	}
\end{UseCase}

\begin{UCtrayectoria}
	
	\UCpaso[\UCactor] En la interfaz de Inventario, hace clic en el botón "Registrar Producto".
	
	\UCpaso[\UCsist] Muestra el formulario "Registrar producto”.

	\UCpaso[\UCsist] Obtiene de la base de datos a los proveedores. [Trayectoria A]
	
	\UCpaso[\UCactor] Llena los campos del formulario.
	
	\UCpaso[\UCactor] Da clic en botón “Confirmar”.

	\UCpaso[\UCsist] Verifica que todos los campos hayan sido llenados [Trayectoria B].

	\UCpaso[\UCsist] Verifica que el nombre del producto sea válido de acuerdo con la regla de negocio RN-051 [Trayectoria C].

	\UCpaso[\UCsist] Verifica que el código de barras sea válido de acuerdo con la regla de negocio RN-052 [Trayectoria D].

	\UCpaso[\UCsist] Verifica que el tamaño de la descripción del producto sea válida de acuerdo con la regla de negocio RN-053 [Trayectoria E].

	\UCpaso[\UCsist] Verifica que el precio de venta ingresado sea válido de acuerdo con la regla de negocio RN-054 [Trayectoria F].

	\UCpaso[\UCsist] Verifica en la base de datos que el código de barras sea único como lo indica la regla de negocios RN-052. [Trayectoria G].
	
	\UCpaso[\UCsist] Guarda en la base de datos lo que se obtuvo del formulario.

	\UCpaso[\UCsist] Muestra el mensaje 1: "Producto registrado exitosamente".

	\UCpaso[] Regresa a la interfaz de inventario.
	
\end{UCtrayectoria}


%...........::::::::::::TRAYECTORIAS ALTERNATIVAS::::::::::::::........
%-------------------------Trayectoria A------------------------------
\begin{UCtrayectoriaA}{A}{No hay proveedores registrados}
	
	\UCpaso[\UCsist] Envía el mensaje 2: "No hay ningún proveedor registrado, registrar uno antes de registrar un producto."

	\UCpaso[\UCactor] Hace clic en "Aceptar".

	\UCpaso[] Regresa a la interfaz de inventario.

\end{UCtrayectoriaA}

%-------------------------Trayectoria B------------------------------
\begin{UCtrayectoriaA}{B}{Faltan campos del formulario por llenar}
	
	\UCpaso[\UCsist] Muestra el mensaje 3: "Debe llenar todos los campos."

	\UCpaso[] Permanece en el formulario.

\end{UCtrayectoriaA}

%-------------------------Trayectoria C------------------------------
\begin{UCtrayectoriaA}{C}{El nombre del producto no es válido}
	
	\UCpaso[\UCsist] Muestra el mensaje 4 al lado del campo "Nombre del producto": "Min.1 / Max. 99 caracteres".

	\UCpaso[] Permanece en el formulario.

\end{UCtrayectoriaA}
%-------------------------Trayectoria D------------------------------
\begin{UCtrayectoriaA}{D}{El código de barras no es válido}
	
	\UCpaso[\UCsist] Muestra el mensaje 5 al lado del campo "Código de barras": "Inválido, revisar la RN5.2"

	\UCpaso[] Permanece en el formulario.

\end{UCtrayectoriaA}
%-------------------------Trayectoria E------------------------------
\begin{UCtrayectoriaA}{E}{El tamaño de la descripción no es válido}
	
	\UCpaso[\UCsist] Muestra el mensaje 6 al lado del campo "Descripción de Producto": "Min.1 / Max. 99 caracteres"

	\UCpaso[] Permanece en el formulario.

\end{UCtrayectoriaA}

%-------------------------Trayectoria F------------------------------
\begin{UCtrayectoriaA}{F}{El precio de venta no es válido}
	
	\UCpaso[\UCsist] Muestra el mensaje 7 al lado del campo "Precio de Venta": "No negativo y 2 decimales"

	\UCpaso[] Permanece en el formulario.

\end{UCtrayectoriaA}

%-------------------------Trayectoria G------------------------------
\begin{UCtrayectoriaA}{G}{Ya hay registrado un producto con el mismo código de barras}
	
	\UCpaso[\UCsist] Envía el mensaje 8: " Ya hay un producto registrado con ese código".
	
	\UCpaso[\UCactor] Hace clic en "Confirmar".

	\UCpaso[\UCsist] Deja de mostrar el mensaje.

	\UCpaso[] Permanece en el formulario.

\end{UCtrayectoriaA}