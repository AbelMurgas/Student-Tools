\begin{UseCase}{CU 1.5.4}{Modificar cantidad de existencias} {
	
	Modificar la cantidad de existencias de un producto por medio de una interfaz, accesible desde la interfaz de Inventario. Por defecto no puede poner números negativos o mayores a dos dígitos.

}
	
\UCitem{Versión}{1.0}
\UCccsection{Administración}
\UCccitem{Autor}{Diego Armando Hernández Penilla}
\UCccitem{Evaluador}{}
\UCccitem{Operación}{}
\UCccitem{Prioridad}{Baja}
\UCccitem{Complejidad}{Media}
\UCccitem{Volatilidad}{Baja}
\UCccitem{Madurez}{Alta}
\UCccitem{Estatus}{Edición}
\UCccitem{Fecha del último estatus}{20 de diciembre del 2021}

%% Copie y pegue este bloque tantas veces como revisiones tenga el caso de uso.
%% Esta sección la debe llenar solo el Revisor
% %--------------------------------------------------------
	%   % Revisión Versión (Anote la versión que se revisó.)
\UCccsection{Revisión Versión 0.1 }
% 	% FECHA: Anote la fecha en que se terminó la revisión.
\UCccitem{Fecha}{} 
% 	% EVALUADOR: Coloque el nombre completo de quien realizó la revisión.
\UCccitem{Evaluador}{}
% 	% RESULTADO: Coloque la palabra que mas se apegue al tipo de acción que el analista debe realizar.
\UCccitem{Resultado}{}
% 	% OBSERVACIONES: Liste los cambios que debe realizar el Analista.
\UCccitem{Observaciones}{
	\begin{UClist}
		%\RCitem{PC1}{\DONE{En la pantalla falta el botón del ojito}}{18 de Abril del 2018}
	\end{UClist}		
}

% %--------------------------------------------------------
	
\UCsection{Atributos}
	\UCitem{Actor(es)}{
		\cdtRef{Actor}{Empleado}
	}
	\UCitem{Propósito}{
		Modificar la cantidad de existencias de un producto.
	}
	\UCitem{Entradas}{
		Existencias.
	}
	\UCitem{Salidas}{
		Mensaje 1: "Las existencias del producto han sido modificadas con éxito",

		Mensaje 2: " No se logró conectar a la base de datos".
	}

	\UCitem{Precondiciones}{
		Haber Registrado al menos un producto con anterioridad.
	}
	\UCitem{Postcondiciones}{
		Ninguna.
	}
	\UCitem{Reglas de negocio}{
		Ninguna.
	}
	\UCitem{Errores}{
		Ninguno
	}
	\UCitem{Tipo}{
		Extend
	}
\end{UseCase}

\begin{UCtrayectoria}
	
	\UCpaso[\UCactor] En la interfaz de Inventario, hace clic en el ícono de flechas encontradas de algún producto por la columna de ajustes.
	
	\UCpaso[\UCsist] Muestra la ventana “Modificar cantidad de existencias”.

	\UCpaso[\UCsist] Muestra los datos del producto a modificar.
 	
	\UCpaso[\UCactor] Modifica el número de existencias.
	
	\UCpaso[\UCactor] Da clic en botón “Confirmar”.

	\UCpaso[\UCsist] Accede a la Base de Datos. [Trayectoria A]

	\UCpaso[\UCsist] Hace los cambios en la Base de Datos.

	\UCpaso[\UCsist] Cierra la ventana “Modificar cantidad de existencias”.

	\UCpaso[\UCsist] Muestra el mensaje 1 en la interfaz de Inventario: "Las existencias del producto han sido modificadas con éxito".

	
\end{UCtrayectoria}

%...........::::::::::::TRAYECTORIAS ALTERNATIVAS::::::::::::::........
%-------------------------Trayectoria A------------------------------
\begin{UCtrayectoriaA}{A}{Sistema no puede acceder a la Base de Datos.}
	
	\UCpaso[\UCsist] Envía el mensaje 2: " No se logró conectar a la base de datos".
	
	\UCpaso[\UCactor] Hace clic en "Confirmar".

	\UCpaso[\UCsist] Deja de mostrar el mensaje.

	\UCpaso[] Permanece en la interfaz Modificar cantidad de existencias.

\end{UCtrayectoriaA}

