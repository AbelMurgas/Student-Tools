%Caso de uso 2
\begin{UseCase} {CU1.1.2.1}{Agregar producto mediante busqueda}{
	El empleado agregará los productos que el cliente desea comprar mediante una busqueda,saber el monto de cada medicamento,monto total y el número de productos agregados a la venta.
}

\UCitem{Versión}{1.0}
\UCccsection{Administración}
\UCccitem{Autor}{Lechuga Canales Héctor Jair}
\UCccitem{Evaluador}{Luis enrique Vázquez}
\UCccitem{Operación}{Ventas}
\UCccitem{Prioridad}{Alta}
\UCccitem{Complejidad}{Media}
\UCccitem{Volatilidad}{Baja}
\UCccitem{Madurez}{Alta}
\UCccitem{Estatus}{Edición}
\UCccitem{Fecha del último estatus}{02 de enero}

% Copie y pegue este bloque tantas veces como revisiones tenga el caso de uso.
% Esta sección la debe llenar solo el Revisor
% --------------------------------------------------------

% Revisión Versión 
% Anote la versión que se revisó
\UCccsection{Revisión Versión 0.1 }

% Fecha
% Anote la fecha en que se terminó la revisión
\UCccitem{Fecha}{03 de enero} 

% Evaluador
% Coloque el nombre completo de quien realizó la revisión
\UCccitem{Evaluador}{Luis enrique Vázquez}

% Resultado
% Coloque la palabra que mas se apegue al tipo de acción que el analista debe realizar
\UCccitem{Resultado}{redacción y ortografía}

% Observaciones
% Liste los cambios que debe realizar el Analista.
\UCccitem{Observaciones}{acentos y signos de puntuación}

% --------------------------------------------------------
	
\UCsection{Atributos}
	\UCitem{Actor(es)}
	{
		\cdtRef{Actor:RT}{Administrador de la farmacia}
	}
	\UCitem{Propósito}
	{
		Agregar productos al carrito de compras mediante una busqueda.
	}
	\UCitem{Entradas}
	{
		
		\UCli Nombre
	}
	\UCitem{Salidas}
	{
		\UCli Código de barras 
		\UCli Descripción del producto
		\UCli Precio de venta
		\UCli Piezas
		\UCli Existencia actual
	}

	\UCitem{Precondiciones}
	{
		\UCli El producto debe tener existencias. 
		\UCli El producto debe estar previamente ingresado en el sistema


	}
	\UCitem{Postcondiciones}
	{
		Ninguna
	}
	\UCitem{Reglas de negocio}
	{
		\cdtRef{RN-3}{Venta con receta médica}
	}
	\UCitem{Errores}
	{
		Ninguna
	}
	\UCitem{Tipo}{}
\end{UseCase}

%Trayectoria principal

\begin{UCtrayectoria}
		
	\UCpaso [\UCactor]Selecciona \cdtButton{Venta} en la interfaz del menú principal. consultar en página \pageref{UI: menu principal}
	\UCpaso [\UCsist]Muestra la interfaz carrito. consultar en página \pageref{UI: carrito}
	\UCpaso [\UCactor]Selecciona el boton \cdtButton{Buscar}.
	\UCpaso [\UCsist]Despliega una pantalla emergente con un input text y un boton \cdtButton{Buscar} en la parte superios izquierda . consultar en página \pageref{UI: carrito buscar}
	\UCpaso [\UCactor]Ingresara el nombre o las primeras letras de este del producto que desea buscar y seleciona \cdtButton{Buscar}.  \refTray{A} \label{CU 1.1.2.1:P5}	
	\UCpaso [\UCsist] Include: \cdtRef{CU 1.5.3}{CU 1.5.3 Consultar productor}Obtiene los datos: Nombre, descripción del producto, cantidad, existencia actual y precio al publico de la BD de Productos de las coincidencias encontradas.
	\UCpaso [\UCactor]Selecciona el producto que desea.
	\UCpaso [\UCsist]Muestra en la parte inferior derecha un boton \cdtButton{Agregar}.
	\UCpaso [\UCactor]Selecciona el boton \cdtButton{Agregar}.
	\UCpaso [\UCsist]Agrega el producto al carrito de compras y regresa a la interfaz carrito. consultar en página \pageref{UI: carrito}

	
\end{UCtrayectoria}


% Trayectorias alternativas
% Trayectoria alternativa A

\begin{UCtrayectoriaA}{A}{El articulo no esta en la BD}

	\UCpaso [\UCsist] se mantiene la pantalla en blanco hasta que se encuetre alguna coincidencia.
	\UCpaso [\UCsist]Regresamos al \ref{CU 1.1.2.1:P5} de la TP.
\end{UCtrayectoriaA}


% Trayectoria alternativa C

% \begin{UCtrayectoriaA}{C}{Inventario con poca existencia.}

% 	\UCpaso [\UCsist]Envía mensaje 2: \cdtRef{MSG}{Pocas existencias}.
% 	\UCpaso [\UCsist]Regresamos al \ref{CU 1.1.1:P6} de la TP.

% \end{UCtrayectoriaA}

% Trayectoria alternativa D

%\begin{UCtrayectoriaA}{D}{Inventario sin existencias.}

%	\UCpaso [\UCsist]envía mensaje 3: \cdtRef{MSG}{ Producto sin existencias}.
%	\UCpaso [\UCsist]Regresamos al \ref{CU 1.1.1:P4} de la TP.

%\end{UCtrayectoriaA}

