\begin{UseCase}{CU3}{Consulta de vacante} { %Mauricio
	
	El sistema muestra al usuario una lista de las vacantes disponibles, de manera filtrada 
muestra estás de acuerdo con lo seleccionado por el usuario. 
}
	
\UCitem{Versión}{1.0}
\UCccsection{Administración}
\UCccitem{Autor}{Reyes Vaca Mauricio Alberto}
\UCccitem{Evaluador}{Jaime}
\UCccitem{Operación}{}
\UCccitem{Prioridad}{Alta}
\UCccitem{Complejidad}{Media}
\UCccitem{Volatilidad}{Baja}
\UCccitem{Madurez}{Alta}
\UCccitem{Estatus}{Edición}
\UCccitem{Fecha del último estatus}{03 de Octubre del 2021}

%% Copie y pegue este bloque tantas veces como revisiones tenga el caso de uso.
%% Esta sección la debe llenar solo el Revisor
% %--------------------------------------------------------
	%   % Revisión Versión (Anote la versión que se revisó.)
\UCccsection{Revisión Versión 0.1 }
% 	% FECHA: Anote la fecha en que se terminó la revisión.
\UCccitem{Fecha}{} 
% 	% EVALUADOR: Coloque el nombre completo de quien realizó la revisión.
\UCccitem{Evaluador}{}
% 	% RESULTADO: Coloque la palabra que mas se apegue al tipo de acción que el analista debe realizar.
\UCccitem{Resultado}{}
% 	% OBSERVACIONES: Liste los cambios que debe realizar el Analista.
\UCccitem{Observaciones}{
	\begin{UClist}
		%\RCitem{PC1}{\DONE{En la pantalla falta el botón del ojito}}{18 de Abril del 2018}
	\end{UClist}		
}

% %--------------------------------------------------------
	
\UCsection{Atributos}
	\UCitem{Actor(es)}{
		\cdtRef{Actor:RT}{Encargado de recursos humanos} y \cdtRef{Actor:AT}{Usuario}
	}
	\UCitem{Propósito}{
		Visualizar la información de las vacantes en existencia.
	}
	\UCitem{Entradas}{
		Ninguna
	}
	\UCitem{Salidas}{
		
		\begin{UClist}
			Información del examen profesional
			\UCli \cdtRef{vacante:nombreVacante}{Vacante}. \ioObtener.
			
			\UCli \cdtRef{vacante:areaVacante}{Area} de la vacante: \ioObtener.
			
			\UCli \cdtRef{vacante:sueldoVacante}{Sueldo} de la vacante: \ioObtener.
			
			\UCli \cdtRef{vacante:descripcionVacante}{Descripcion} de la vacante: \ioObtener.				
		
			\UCli \cdtRef{vacante:turnoVacante}{Turno} de la vacante: \ioObtener.

		\end{UClist}
	}

	\UCitem{Precondiciones}{
		Haber registrado exitosamente a por lo menos una vacante con la opción “Crear 
vacante”.
	}
	\UCitem{Postcondiciones}{
		Ninguna
	}
	\UCitem{Reglas de negocio}{
		Ninguna
	}
	\UCitem{Errores}{
		Ninguna
	}
	\UCitem{Tipo}{Secundario, extiende del caso de uso \cdtIdRef{CUTI5.1-2}{Registrar vacantes}.}
\end{UseCase}

\begin{UCtrayectoria}
	
	\UCpaso[\UCactor] Selecciona la opción “Bolsa de trabajo”.
	
	\UCpaso[\UCsist]  Muestra un selector de opciones llamado “puesto”.
	
	\UCpaso[\UCsist] Selecciona la opción que prefiera buscar. 
	
	\UCpaso[\UCsist] Selecciona el botón “Buscar”.
	
	\UCpaso[\UCsist] Muestra la pantalla \cdtIdRef{IUTI5.1-8}{Visualizar Resultado} con la información obtenida.
	
	\UCpaso[\UCactor] Solcita terminar la consulta presionando el botón \cdtButton{Regresar}. \label{CUTI5.1-8:terminaConsulta}
	
	\UCpaso[\UCsist] Muestra la pantalla \cdtIdRef{IUTI5.1-2}{Index}.
	
\end{UCtrayectoria}