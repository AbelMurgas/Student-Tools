\begin{UseCase}{CU 1.5.3}{Modificar Datos del Producto} {
	
	Modificar los datos de un producto registrado en el inventario a partir de un formulario idéntico a aquel que se utilizó para registrarlo en primer lugar.

}
	
\UCitem{Versión}{1.0}
\UCccsection{Administración}
\UCccitem{Autor}{Luis Carlos Pimentel Escobedo}
\UCccitem{Evaluador}{}
\UCccitem{Operación}{}
\UCccitem{Prioridad}{Alta}
\UCccitem{Complejidad}{Media}
\UCccitem{Volatilidad}{Baja}
\UCccitem{Madurez}{Alta}
\UCccitem{Estatus}{Edición}
\UCccitem{Fecha del último estatus}{20 de diciembre del 2021}

%% Copie y pegue este bloque tantas veces como revisiones tenga el caso de uso.
%% Esta sección la debe llenar solo el Revisor
% %--------------------------------------------------------
	%   % Revisión Versión (Anote la versión que se revisó.)
\UCccsection{Revisión Versión 0.1 }
% 	% FECHA: Anote la fecha en que se terminó la revisión.
\UCccitem{Fecha}{} 
% 	% EVALUADOR: Coloque el nombre completo de quien realizó la revisión.
\UCccitem{Evaluador}{}
% 	% RESULTADO: Coloque la palabra que mas se apegue al tipo de acción que el analista debe realizar.
\UCccitem{Resultado}{}
% 	% OBSERVACIONES: Liste los cambios que debe realizar el Analista.
\UCccitem{Observaciones}{
	\begin{UClist}
		%\RCitem{PC1}{\DONE{En la pantalla falta el botón del ojito}}{18 de Abril del 2018}
	\end{UClist}		
}

% %--------------------------------------------------------
	
\UCsection{Atributos}
	\UCitem{Actor(es)}{
		\cdtRef{Actor}{Empleado}
	}
	\UCitem{Propósito}{
		Modificar los datos de un producto  ya registrado.
	}
	\UCitem{Entradas}{
		Nombre del producto,
		Proveedor,
		Tipo,
		Precio de Compra,
		Precio de Venta,
		Mínimo de Existencias,
	}
	\UCitem{Salidas}{
		Mensaje 1: "El producto ha sido registrado con éxito.",
		Mensaje 2: "Debe llenar todos los campos.",
		Mensaje 3: "MIN.1 carácter. MAX. 99.",
		Mensaje 4: "MAX. 3 enteros, 2 decimales.",
		Mensaje 5: "Introduce un valor mayor o igual a cero.",
		Mensaje 6:  " No se logró conectar a la base de datos.",
		Mensaje 7: " Ya hay registrado un producto con los mismos datos".
	}

	\UCitem{Precondiciones}{
		Haber Registrado al menos un producto con anterioridad.
	}
	\UCitem{Postcondiciones}{
		Ninguna.
	}
	\UCitem{Reglas de negocio}{
		RN5.2.- El nombre del producto no debe exceder los 99 caracteres ni ser menor a 1 carácter,
		RN5.3.- El precio de cada producto es un número mayor a cero, con máximo dos dígitos después del punto decimal y no mayor a 5 digitos (incluyendo a los decimales),
		RN5.4.- La cantidad mínima de existencias debe ser mayor o igual a cero y no exceder los 2 dígitos,
		RN5.5.- No puede haber dos productos con los mismos datos
	}
	\UCitem{Errores}{
		Faltan campos del formulario por llenar,
		El nombre del producto no es válido,
		El formato del precio de compra no es válido,
		El formato del precio de venta no es válido,
		La cantidad de existencias mínima no es válida,
		Sistema no puede acceder a la Base de Datos.
		Hay registrado otro producto con los mismos datos.
	}
	\UCitem{Tipo}{
		Extend
	}
\end{UseCase}

\begin{UCtrayectoria}
	
	\UCpaso[\UCactor] En la interfaz de Inventario, hace clic en el ícono de engrane de algún producto por la columna de ajustes.
	
	\UCpaso[\UCsist] Muestra el formulario “Modificar producto”.

	\UCpaso[\UCsist] 
	
	\UCpaso[\UCactor] Modifica los campos del formulario.

	\UCpaso[] Include 1:  C.U.1.5.7 "Seleccionar Proveedor para Producto".
	
	\UCpaso[\UCactor] Da clic en botón “Confirmar”.

	\UCpaso[\UCsist] Verifica que todos los campos hayan sido llenados [Trayectoria A].

	\UCpaso[\UCsist] Verifica que el nombre del producto sea válido de acuerdo con la regla de negocio RN5.2 [Trayectoria B].

	\UCpaso[\UCsist] Verifica que el precio de compra ingresado sea válido de acuerdo con la regla de negocio RN5.3 [Trayectoria C].

	\UCpaso[\UCsist] Verifica que el precio de venta ingresado sea válido de acuerdo con la regla de negocio RN5.3 [Trayectoria D].
	
	\UCpaso[\UCsist] Verifica que la cantidad del producto sea válida de acuerdo con la regla de negocio RN5.4 [Trayectoria E].

	\UCpaso[\UCsist] Se conecta a la base de datos. [Trayectoria F]

	\UCpaso[\UCsist] Verifica en la base de datos que los datos ingresados cumplan la regla de negocios RN5.5. [Trayectoria G].

	\UCpaso[\UCsist] Realiza una update en la base de datos con los nuevos datos.

	\UCpaso[\UCsist] Muestra el mensaje 1: "El producto ha sido modificado con éxito".

	\UCpaso[\UCactor] Da clic en botón “Confirmar”.

	\UCpaso[\UCsist] Muestra a la interfaz de Inventario.

	\UCpaso[\UCsist] Obtiene el producto que acaba de modificar de la Base de Datos.

	\UCpaso[\UCsist] Muestra producto que acaba de modificar.
	
\end{UCtrayectoria}

%...........::::::::::::TRAYECTORIAS ALTERNATIVAS::::::::::::::........
%-------------------------Trayectoria A------------------------------
\begin{UCtrayectoriaA}{A}{Faltan campos del formulario por llenar}
	
	\UCpaso[\UCsist] Envía el mensaje 2: "Debe llenar todos los campos."

	\UCpaso[] Permanece en el formulario Datos del Producto.

\end{UCtrayectoriaA}
%-------------------------Trayectoria B------------------------------
\begin{UCtrayectoriaA}{B}{El nombre del producto no es válido}
	
	\UCpaso[\UCsist] Muestra el mensaje 3 al lado del campo "Nombre del producto": "MIN.1 carácter. MAX. 99"

	\UCpaso[] Permanece en el formulario Datos del Producto.

\end{UCtrayectoriaA}
%-------------------------Trayectoria C------------------------------
\begin{UCtrayectoriaA}{C}{El formato del precio de compra no es válido}
	
	\UCpaso[\UCsist] Muestra el mensaje 4 al lado del campo "Precio de Compra": "MAX. 3 enteros, 2 decimales"

	\UCpaso[] Permanece en el formulario Datos del Producto.

\end{UCtrayectoriaA}
%-------------------------Trayectoria D------------------------------
\begin{UCtrayectoriaA}{D}{El formato del precio de venta no es válido}
	
	\UCpaso[\UCsist] Envía el mensaje 4 al lado del campo "Precio de Venta": "MAX. 3 enteros, 2 decimales".

	\UCpaso[] Permanece en el formulario Datos del Producto.

\end{UCtrayectoriaA}
%-------------------------Trayectoria E------------------------------
\begin{UCtrayectoriaA}{E}{La cantidad de existencias mínima no es válida}
	
	\UCpaso[\UCsist] Envía el mensaje 5.- "Introduce un valor mayor o igual a cero".	

	\UCpaso[] Permanece en el formulario Datos del Producto.

\end{UCtrayectoriaA}

%-------------------------Trayectoria F------------------------------
\begin{UCtrayectoriaA}{F}{Sistema no puede acceder a la Base de Datos.}
	
	\UCpaso[\UCsist] Envía el mensaje 6: " No se logró conectar a la base de datos".
	
	\UCpaso[\UCactor] Hace clic en "Confirmar".

	\UCpaso[\UCsist] Deja de mostrar el mensaje.

	\UCpaso[] Permanece en el formulario Datos del Producto.

\end{UCtrayectoriaA}
%-------------------------Trayectoria G------------------------------
\begin{UCtrayectoriaA}{G}{Hay registrado otro producto con los mismos datos}
	
	\UCpaso[\UCsist] Envía el mensaje 7: " Ya hay registrado un producto con los mismos datos".
	
	\UCpaso[\UCactor] Hace clic en "Confirmar".

	\UCpaso[\UCsist] Deja de mostrar el mensaje.

	\UCpaso[] Permanece en el formulario Datos del Producto.

\end{UCtrayectoriaA}

